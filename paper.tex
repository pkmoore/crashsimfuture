p
\documentclass[twocolumn]{article}

\begin{document}

All of these steps will be taken with the goal of improving the tool by
increasing its usage.

\section{Future-proofing the tool}

One aspect of ensuring a tool or technique has a long and successful
lifespan is ensuring that it uses up to date technologies and follows
current best practices for the communities with which it is intended to
work.



\subsection{Updating components}

In the short term, two tasks to be undertaken are updating the tool to
remove its dependence on python 2 and to support 64-bit, x86-64
applications.

64-bit support
support new version of python

\subsection{Supporting new platforms}

While CrashSimulator has been shown to be successful at finding
environmental bugs when limited to only operating on Linux applications
there is a great deal of opportunity to find more interesting, impactful
bugs by examining the environmental differences resulting from applications
running on completely different operating systems.  As a result, improving
CrashSimulator to run on Windows and OS X is a major future goal of this
work.

Key to realizing this support is dealing with the fact that each operating
system provides different system calls.  For example, Linux uses {\tt
open()} to open files while Windows provides {\tt NtOpenFile()}.  In their
current form, checkers and mutators describe anomalies and expected
application behavior in terms of Linux system calls.  Addressing this
limitation means elevating this description to a higher level and
implementing a layer to translate this description down to the actual
system calls a given operating system provides.

An additional area for exploration is determining whether or not a given
anomaly is able to be simulated on a given operating system.  If an anomaly
only appears in the presence of a operating system feature only present on
a specific operating system, it would be an error to simulate it on other
operating systems.  Classifying anomalies in terms of where it makes sense
to simulate them would reduce the number of false positive reports the tool
could make.

\subsection{Grouping Anomalies by Operating System}

Addressing the previously mentioned limitations will provide CrashSimulator
with an interesting new capability: the ability to simulate the experience
of running an application on a different operating system.  This can be
accomplished by grouping anomalies into sets representing each of the
operating systems on which an application is expected to run. For example,
the corpus of anomalies included with this future version of CrashSimulator
could be segregated into groups like those present on Windows 10, those
present on Ubuntu 18.04 and so on.  Once grouped in this fashion,
applications could be tested against a chosen group in order to see how it
would perform on the represented operating system without having to go
through of the effort of deploying it there.


\subsection{Porting new features to dependencies}

One of the major technical hurdles to implementing CrashSimulator was our
modification of {\tt rr} to allow the output of strace-style recordings and
to allow process set cloning.  A future goal of this work is to provide
these features back to the {\tt rr} project.  Doing so would benefit {\tt
rr's} audience by allowing them acccess to these features
and would help CrashSimulator by ensuring that functionality on which it
depends will continue to exist as {\tt rr} continues to grow and improve.


\section{Usability Improvements}

The primary developer effort associated with using CrashSimulator is in
constructing the checkers and mutators that drive its testing process.  In
much the same vein as a constructing a unit test, the analysis effort
required to produce these artifacts is closely tied to an individual user's
skill and difficult to improve through technical means.  However, we
believe the effort of actually writing a checker and mutator after the
analysis is complete could be dramatically reduced.  Much of the complexity
associated with constructing these elements currently comes down to their
being implemented in fully featured python.  We believe this work could be
reduced through the use of a simpler checker and mutator description language
description language.  As we have discussed in more detail
elsewhere in this document, our plans consists of a language akin to
regular expressions focusing on sequences of system calls and their
arguments rather than the contents of strings.  Proof of concept
experiments with this design have shown promise both in terms of
feasibility of implementation and in real benefits to the anomaly encoding
process.  As a result, we believe this approach has the potential to reduce
both the size and complexity of the code behind CrashSimulator's checkers
and mutators.

\subsection{ Safe sandbox in which user submitted}

As we will discuss in later sections, one of our main goals is to have a
CrashSimulator's community contribute to the corpus of anomalies the tool
can simulate.  In order to do this safely, the tool needs to be able to
take code from unknown, potential untrusted developers and execute it
without the possibility of negative effects, whether unintentional or
malicious, on the host system.  Our approach to offering this capability is
to execute checkers and mutators written in our description language in a
sandbox where they may only read the stream of system calls
from the application being tested and respond with either the {\tt accept} or
{\tt reject} result of the state machines they contain.  These limitations
dramatically reduce the ability of third-party checkers and mutators to
interact with the host system in unintended ways.
Our proposed sandbox could also offer further protection by enforcing
limitations on the checkers and mutators running within it.  For example,
it may be useful to impose memory or processing limits in order to prevent
long running or poorly implemented checkers and mutators from harming the
stability of host system.  These new capabilities would both improve the
tool's safety and ease of use on the small scale and help speed
community building by reducing the effort required to vet
community submissions.


\section{Community building}

The community that grows around a tool or technique is likely an even
bigger contributor to its long term success than its technical merits.
This is especially true in the case of SEA and CrashSimulator due to the
way the tool and technique become more effective as users contribute to the
corpus of anomalies that can be simulated.  As a result, developing a
vibrant community is a paramount future goal of this work.

The technical side of this goal has two parts.  The first is to establish
an online repository into which users can submit checkers and mutators they
have written so that others may download and take advantage of them.  The
second part is to set up the infrastructure required to allow
CrashSimulator to be integrated into continuous integration and automated
testing products in use by major open source applications.  Each of these
parts will ease the process of adopting CrashSimulator and increase the
audience to which the tool has been introduced.

\subsection{Establishing an Anomaly Repository}

One of the primary advantages offered by SEA and CrashSimulator is the way
their effectiveness and capabilities can be augmented by the addition of
new anomalies.  To best take advantage of this fact it makes sense to
provide to users as many anomalies as is feasible in an easily accessible
fashion.  In the simplest case, this involves supplying a set of
straightforward anomalies that are known to be useful in testing a wide
variety of applications directly with the tool when it is installed.
However, as the number of anomalies being contributed grows, this approach
will not scale.  Shipping a large number of anomalies, including some of
which will likely be useful only in specific situations, will harm the
tool's performance and user experience.  To address this problem, we
propose establishing an online anomaly repository.  This approach would
allow us to  store and provide as many anomalies as possible while avoiding
the aforementioned problems.

In addition to avoiding performance and user experience problems, an online
anomaly repository can act as a centralized location for anomaly metadata
that can dramatically improve their usefulness. For example, such a
repository would provide a location where anomalies could be tagged and
categorized along dimensions like the operating systems they can appear
involve, their performance characteristics, and the sorts of applications
with which they are likely to be useful.  Additionally, it would also
provide a centralized location for community driven anomaly documentation
and discussion.  This information would allow users to be informed enough
to download the individual anomalies or sets of anomalies that are most
useful for their specific use case.


\subsection{Integration into CI platforms}

A second thrust for fostering community expansion is setting up the
infrastructure to allow CrashSimualtor to be integrated into existing
continuous integration platforms.  Continuous integration platforms, such
as TravisCI and CircleCI, are a popular solution for open source projects
seeking to ensure their applications are continually and thoroughly tested
as new code contributions are submitted.  Given the ``universal'' nature of
CrashSimulator's testing process, the tool is a perfect fit for this
scenario.  Our goal is to provide a configuration of CrashSimulator
suitable for unattended execution to which test jobs can be submitted
automatically as part of the continuous integration pipeline.  Once create,
this configuration would allow for testing with CrashSimulator and a
pre-selected set of anomalies by simply checking a box in the projects
testing configuration.

We believe this use case could drive rapid growth in CrashSimulator's
community because it removes most of the friction associated with adopting
a new tool.  Once the ``automated'' version of the tool is adopted it is a
natural next step for users to move into more advanced use cases and,
ideally, contribution to the tool as their needs expand.  Further, by
integrating with exsting continuous integration infrastructure,
CrashSimulator gains access to those platforms' communities resulting in
exposure to users who might not have otherwise come into contact with the
tool.  The end result of all this is a new, expanded, set of community
members that can contribute to the tool and improve its efficacy for
everyone.

\subsection{Documentation on identifying and capturing anomalies}

\end{document}


%  Product evangelism
%  User study
%  Wider instruction on selecting and capturing anomalies
