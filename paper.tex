
\documentclass{article}

\begin{document}

All of these steps will be taken with the goal of improving the tool by
increasing its usage.

\section{Future-proofing the tool}

\subsection{Updating components}
64-bit support
support new version of python

\subsection{Supporting new platforms}

While CrashSimulator has been shown to be successful at finding
environmental bugs when limited to only operating on Linux applications
there is a great deal of opportunity to find more interesting, impactful
bugs by examining the environmental differences resulting from applications
running on completely different operating systems.  As a result, improving
CrashSimulator to run on Windows and OS X is a major future goal of this
work.

Key to realizing this support is dealing with the fact that each operating
system provides different system calls.  For example, Linux uses {\tt
open()} to open files while Windows provides {\tt NtOpenFile()}.  In their
current form, checkers and mutators describe anomalies and expected
application behavior in terms of Linux system calls.  Addressing this
limitation means elevating this description to a higher level and
implementing a layer to translate this description down to the actual
system calls a given operating system provides.

An additional area for exploration is determining whether or not a given
anomaly is able to be simulated on a given operating system.  If an anomaly
only appears in the presence of a operating system feature only present on
a specific operating system, it would be an error to simulate it on other
operating systems.  Classifying anomalies in terms of where it makes sense
to simulate them would reduce the number of false positive reports the tool
could make.

\subsection{Grouping Anomalies by Operating System}

Addressing the previously mentioned limitations will provide CrashSimulator
with an interesting new capability: the ability to simulate the experience
of running an application on a different operating system.  This can be
accomplished by grouping anomalies into sets representing each of the
operating systems on which an application is expected to run. For example,
the corpus of anomalies included with this future version of CrashSimulator
could be segregated into groups like those present on Windows 10, those
present on Ubuntu 18.04 and so on.  Once grouped in this fashion,
applications could be tested against a chosen group in order to see how it
would perform on the represented operating system without having to go
through of the effort of deploying it there.


\subsection{Porting new features to dependencies}

One of the major technical hurdles to implementing CrashSimulator was our
modification of {\tt rr} to allow the output of strace-style recordings and
to allow process set cloning.  A future goal of this work is to provide
these features back to the {\tt rr} project.  Doing so would benefit {\tt
rr's} audience by allowing them acccess to these features
and would help CrashSimulator by ensuring that functionality on which it
depends will continue to exist as {\tt rr} continues to grow and improve.


\section{Usability Improvements}
  Domain specific language for describing checkers and mutators


\section{Community building}
  Product evangelism
  User study
  Wider instruction on selecting and capturing anomalies
  CI platform support

\end{document}
