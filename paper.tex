\documentclass[twocolumn]{article}

\begin{document}

The ultimate goal of our future work on SEA and CrashSimulator is to build
a large, self-sustaining community of users.  To achieve this we have a
two-part plan that focuses on both improving the tool and making specific
community building efforts.  In this section we detail what specifically
needs to be done to accomplish both parts of this plan.


\section{Tool Improvements and New Features}

To make CrashSimulator more appealing to our target users we propose taking
the following actions:


\subsection{Supporting new platforms}

CrashSimulator has been shown to be successful at finding
environmental bugs in Linux applications. While this is useful,
there is a great deal of opportunity to find more interesting, impactful
bugs by examining the environmental differences resulting from applications
running on completely different operating systems.
In order to find these
bugs, CrashSimulator must be improved so that it can run on, and simulate
anomalies present in, these additional platforms.  Initially we plan to
support Windows and OS X.

Key to realizing this support is dealing with the fact that each operating
system provides different system calls.  For example, Linux uses {\tt
open()} to open files while Windows provides {\tt NtOpenFile()}.  In their
current form, checkers and mutators describe anomalies and expected
application behavior in terms of Linux system calls.  Addressing this
limitation means elevating this description to a higher level and
implementing a layer to translate this description down to the actual
system calls a given operating system provides.  Building upon the previous
example, a higher level description in a mutator may specify an operation
like ``open file'' which is translated to either {\tt open()} or {\tt
NtOpenFile()} as is appropriate.

An additional area for exploration is determining whether or not a given
anomaly is able to be simulated on a given operating system.  If an anomaly
only appears in the presence of a operating system feature , it would be an
error to simulate it on operating systems where this feature is not
present.  Classifying anomalies in terms of where it makes sense to
simulate them would reduce the number of false positive reports the tool
could make.


\subsection{Grouping Anomalies by Operating System}

Adding support for new operating systems will provide CrashSimulator with
an interesting new capability: the ability to simulate the experience of
running an application on a different operating system.  This can be
accomplished by grouping anomalies into sets representing each of the
operating systems on which an application is expected to run. For example,
the corpus of anomalies included with a future version of CrashSimulator
could be segregated into groups such as those present on Windows 10, those
present on Ubuntu 18.04 and so on.  Once grouped in this fashion,
applications could be tested against a chosen group in order to see how it
would perform on the represented operating system without. In this way,
CrashSimualtor could offer insights into how an application will perform on
a given operating system without having to go through of the effort of
deploying it.


\section{Usability Improvements}

Other improvements we propose will directly benefit CrashSimulator's
usability.  The primary developer effort associated with using
CrashSimulator is in constructing the checkers and mutators that drive its
testing process.  In much the same vein as a constructing a unit test, the
analysis effort required to produce these artifacts is closely tied to an
individual user's skill and difficult to improve through technical means.
However, we believe the effort of actually writing a checker and mutator
after the analysis is complete could be dramatically reduced.  Much of the
complexity associated with constructing these elements currently comes down
to their being implemented in fully featured python.  We believe this work
could be reduced through the use of a simpler checker and mutator
description language description language.  As we have discussed in more
detail elsewhere in this document, our plans consists of a language akin to
regular expressions focusing on sequences of system calls and their
arguments rather than the contents of strings.  Proof of concept
experiments with this design have shown promise both in terms of
feasibility of implementation and in real benefits to the anomaly encoding
process.  As a result, we believe this approach has the potential to reduce
both the size and complexity of the code behind CrashSimulator's checkers
and mutators.


\subsection{Sandboxing}

As we will discuss in later sections, one of our main goals is to have a
CrashSimulator's community contribute to the corpus of anomalies the tool
can simulate.  In order to do this safely, the tool needs to be able to
take code from unknown, potential untrusted developers and execute it
without the possibility of negative effects, whether unintentional or
malicious, on the host system.  Our approach to offering this capability is
to execute checkers and mutators written in our description language in a
sandbox where they may only read the stream of system calls
from the application being tested and respond with either the {\tt accept} or
{\tt reject} result of the state machines they contain.  These limitations
dramatically reduce the ability of third-party checkers and mutators to
interact with the host system in unintended ways.
Our proposed sandbox could also offer further protection by enforcing
limitations on the checkers and mutators running within it.  For example,
it may be useful to impose memory or processing limits in order to prevent
long running or poorly implemented checkers and mutators from harming the
stability of host system.  These new capabilities would both improve the
tool's safety and ease of use on the small scale and help speed
community building by reducing the effort required to vet
community submissions.


\subsection{Updating components}

In addition to new improvements, CrashSimulator requires some updates to
replace soon-to-be unsupported components.  While it doesn't outright stop
the usage of a tool, relying on deprecated components introduces friction
into the process of getting it adopted and accepted into software
distribution channels and repositories.  In CrashSimulator's case, there
are two dependencies that need to be updated.  The first of these is
porting pieces of the code base written in Python 2 to Python 3.  This must
be done because Python 2 is no longer supported as of January of 2020.
While the code will continue to work, relying on an unsupported version of
Python could cause challenges with getting the tool accepted by some Linux
distributions.  Additionally, there is concern in the Python community
around writing code using an unsupported version -- a fact that could turn
away potential CrashSimulator users.  While, porting a large software
project is not to be undertaken lightly, there have been a great deal of
large software projects that have accomplished the move from Python 2 to 3
successfully.  As such, documentation and best practices for doing so are
abundantly available.

The second dependency that requires updating is the processor architecture
that CrashSimulator supports.  Because the tool performs its simulation
work manipulating the memory and register values that contain the results
and side effects of system calls, the initial version of the tool was
constrained to only supporting x86, 32-bit systems in order to development
time.  At this point, 64-bit systems (typically x86\_64) are the norm and
many Linux distributions are reducing the effort put into their 32-bit
flavors.  While it is possible to use CrashSimulator to test applications
by re-compiling them for 32-bit x86, this reduces two of the tool's
advantages -- the ``universal'' nature of its tests and its ability to test
an application without its source code.  In order to avoid a cumbersome
situation where is users must recompile their applications, CrashSimulator
must be improved to support applications compiled for x86\_64.


\section{Community Focused Improvements}

As was mentioned previously, our main goal is to foster a community around
CrashSimulator and SEA.  The community that grows around a tool or
technique is likely an even bigger contributor to its long term success
than its technical merits.  This is especially true in the case of SEA and
CrashSimulator due to the way the tool and technique become more effective
as users contribute to the corpus of anomalies that can be simulated.  The
external work we propose to achieve this end has two parts.  The first is
to establish an online repository into which users can submit checkers and
mutators they have written so that others may download and take advantage
of them.  The second part is to set up the infrastructure required to allow
CrashSimulator to be integrated into continuous integration and automated
testing products in use by major open source applications.  Each of these
parts will ease the process of adopting CrashSimulator and increase the
audience to which the tool has been introduced.

\subsection{Establishing an Anomaly Repository}

One of the primary advantages offered by SEA and CrashSimulator is the way
their effectiveness and capabilities can be augmented by the addition of
new anomalies.  To best take advantage of this fact it makes sense to
provide to users as many anomalies as is feasible in an easily accessible
fashion.  In the simplest case, this involves supplying a set of
straightforward anomalies that are known to be useful in testing a wide
variety of applications directly with the tool when it is installed.
However, as the number of anomalies being contributed grows, this approach
will not scale.  Shipping a large number of anomalies, including some of
which will likely be useful only in specific situations, will harm the
tool's performance and user experience.  To address this problem, we
propose establishing an online anomaly repository.

An online anomaly repository can act as a centralized location for anomaly
metadata such as
the operating systems on which they appear,
their performance characteristics, and the sorts of applications
with which they are likely to be useful.  Additionally, it could also
act as centralized location for community driven anomaly documentation
and discussion.  This information would allow users to be informed enough
to download the individual anomalies or sets of anomalies that are most
useful for their specific use case.


\subsection{Integration into CI platforms}

A second community-oriented improvement is building the infrastructure
necessary to allow CrashSimualtor to interoperate with existing continuous
integration platforms.  Continuous integration platforms, such as TravisCI
and CircleCI, are a popular solution for open source projects seeking to
ensure their applications are continually and thoroughly tested as new code
contributions are submitted.  Given the ``universal'' nature of
CrashSimulator's testing process, the tool is a perfect fit for this
scenario.  Our goal is to provide a configuration of CrashSimulator
suitable for unattended execution to which test jobs can be submitted
automatically as part of a continuous integration pipeline.  Once created,
this configuration would allow for testing with a
pre-selected set of anomalies by simply checking a box in the projects
testing configuration.

We believe this setup could drive rapid growth in CrashSimulator's
community because it removes most of the friction associated with adopting
a new tool.  Once the ``automated'' version of the tool is in place it is a
natural next step for users to move into more advanced use cases and,
ideally, contribution to the tool as their needs expand.  Further, by
integrating with existing continuous integration infrastructure,
CrashSimulator gains access to those platforms' communities resulting in
exposure to users who might not have otherwise come into contact with the
tool.  The end result of all this is a new, expanded, set of community
members that can contribute to the tool and improve its efficacy for
everyone.


\subsection{Giving New Features Back to the Community}

Part of the effort of developing CrashSimulator was making modifications to
the {\tt rr} record-and-replay debugger.  Specifically, these changes
allowed for two new features: the output of strace-style recordings and the
ability to clone process sets as discussed in the SEA paper.  We propose
giving these features back to the {\tt rr} community by porting them
into main-line {\tt rr}.  Doing so would benefit {\tt rr's
users by allowing them access to these new capabilities and would help
CrashSimulator by ensuring that functionality on which it depends will
continue to exist as {\tt rr} continues to grow and improve.


\end{document}


%  Documentation on identifying and capturing anomalies
%  Product evangelism
%  User study
%  Wider instruction on selecting and capturing anomalies
